\documentclass[../../main.tex]{subfiles}
\usepackage{amsmath,amsthm,amsfonts,latexsym,mathtools,bm,ulem,amssymb,tikz,circuitikz,graphicx}
\usepackage{times}
\usepackage[subrefformat=parens]{subcaption}
\usepackage{here}
\usepackage{siunitx}
\usepackage{physics}
\usepackage{mhchem}
\usepackage{upgreek}%ギリシャ文字を立てるのだ
\numberwithin{equation}{section}
\numberwithin{table}{section}
\numberwithin{figure}{section}
\usepackage{wrapfig}%文中に画像を入れる
\begin{document}
\section{フィードバック}
  \textgt{前期レポートで先生からもらったフィードバックを一応残してます。}


  この報告書について、先生からいただいたフィードバックをまとめる。次回以降のレポートを書くときに気をつける事項を箇条書きにする。
  \subsection{概要}
    \begin{itemize}
      \item 数値を示す際、$0.023\sim0.137$のように$\sim$ではなく$-$を使うのが一般的
      \item 「$p_0\cos^{p_1}\theta+p_2$でフィッティングを行い」など、フィッティング関数のどの部分がフィッティングパラメータなのか明記すべき
    \end{itemize}
  \subsection{理論}
    \begin{itemize}
      \item 数値と単位の間のスペース忘れないようにしよう
      \begin{itemize}
        \item[$\Rightarrow$] 開ける大きさは半角スペースひとつ分であり、LaTeXの数式モード内の「\textbackslash,」は半角スペースの半分だけしか開かないため注意。今まで「\textbackslash,」が半角スペースと同じ大きさだと勘違いしていた。(参考: 図\ref{fig:tex-space})
        
        図は省略
      \end{itemize}
      \item 他の文献からの引用をするときは\cite{PDG:ptm}のように書いて、最後のセクションなどに引用元をまとめて記すのが良い
      \begin{itemize}
        \item[$\Rightarrow$] thebibliography環境で管理する試みをしてみた。subfileシステムとも互換性も確認した。
      \end{itemize}
      \item 物理定数、例えば光速度などを$c$などと表すときはそれを宣言すること
      \item (全体的に)口語にならないように意識する
      \item 議論の流れから外れるサブ議論をtcolorboxなどを使ってコラム的に書くことは報告書には相応しくない
    \end{itemize}
  \subsection{実験1}
    \begin{itemize}
      \item 表のキャプションは表の上に配置するのがルール
    \end{itemize}
  \subsection{実験3.5}
    \begin{itemize}
      \item itemize環境を複数個使って、入れ子構造にして配置するのは思考の整理には便利であるが、報告書としてはふさわしくない。文章として書く。
    \end{itemize}
  \subsection{付録}
    \begin{itemize}
      \item 付録の前に全体のまとめ(結論)や今後の展望があると良い
    \end{itemize}
  
\end{document}