\documentclass[../../main.tex]{subfiles}
\usepackage{amsmath,amsthm,amsfonts,latexsym,mathtools,bm,ulem,amssymb,tikz,circuitikz,graphicx}
\usepackage{times}
\usepackage[subrefformat=parens]{subcaption}
\usepackage{here}
\usepackage{siunitx}
\usepackage{physics}
\usepackage{mhchem}
\usepackage{upgreek}%ギリシャ文字を立てるのだ
\numberwithin{equation}{section}
\numberwithin{table}{section}
\numberwithin{figure}{section}
\usepackage{wrapfig}%文中に画像を入れる
\begin{document}
\section{付録}
  ローカルすぎる学習内容をこの章に記す。
  \subsection{オシロスコープDAQの作業}
    \begin{itemize}
      \item \textgt{LANケーブルが刺さっているか確認する}
      \item PC側でTCP/IP→IPv4の設定→DHCP手動→IPアドレス(自機)"192.168.10.$\sim$"、サブネットマスク"255.255.255.0"
      \item オシロスコープ側で"192.168.10.10"
      \item (PCからping飛ばしてみる)
      \item run.shのあるディレクトリで"./run.sh \%d"(詳細はDAQコードのコマンドライン引数を確認してみるとか、シェルスクリプト側を書き換えてみるなど...)
    \end{itemize}
	\subsection{過去の卒研の記述}
    『空気シャワーの観測』によると空気シャワーの広がりを調べたそう。それを参考にすると空気シャワーの広がりってめっちゃ小さい... consistentかどうかは追って調べる必要がありそう。
\end{document}